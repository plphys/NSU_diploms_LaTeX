\chapter{Об этом шаблоне}\label{ch:about}




Данный шаблон создан на основе \href{https://docs.cntd.ru/document/1200026224}{ГОСТ 7.32-2001} ''Отчет о научно-иссле\-до\-ва\-тельской работе. Структура и правила оформления'' и рекомендаций деканата ФФ НГУ:

\href{http://www.phys.nsu.ru/main/index.php/main/news/652-2016-03-01-news-kursovye}{\small http://www.phys.nsu.ru/main/index.php/main/news/652-2016-03-01-news-kursovye}

Настоятельно рекомендуется ознакомиться с этими двумя ресурсами.

Шаблон предназначен для вёрстки посредством \verb*|pdflatex| или \verb*|xelatex| и \verb*|bibtex| (библиография)

Необходимые файлы шаблона:
\begin{enumerate}
	\item \verb*|kurs3.cls| --- настройки класса: объявление вспомогательных команд, настройка оформления и титульного листа;
	\item \verb*|Preamble.tex| --- файл с перечислением подключаемых пакетов;
	\item \verb*|ugost2008.bst| --- стилевой пакет библиографии по ГОСТу 7.0.5-2008. Необходим при вёрстке на локальном ПК --- при использовании \href{https://www.overleaf.com}{overleaf} не нужен. Он там есть по-умолчанию.
	\item \verb*|kurs3temp.tex| --- основной документ, в который подключаются остальные файлы и сам текст;
	\item  \verb*|ref.bib| --- файл с библиографией в формате \verb*|bibtex|'а.
\end{enumerate}

При вёрстке \verb*|pdflatex|'ом шаблон по-умолчанию использует шрифт с засечками Computer Modern Roman (CMR). Это не Times New Roman (TMN) (проприетарный шрифт), который указывается в различных требованиях и ГОСТах. Отдельный вопрос почему он там указывается. Скорее всего фактически имелся ввиду ''шрифт с засечками''. Во времена написания требований офисный документооборот осуществлялся с помощью ПО Microsoft, в котором этот шрифт был основным шрифтом с засечками. На практике проблем с использованием CMR обычно не бывает (особенно в курсовых работах).

Если всенепременно хотите использовать TMN, то есть два пути:

\begin{enumerate}
	\item Если использовать \verb*|pdflatex|, то нужно в файле  \verb*|kurs3.cls| раскомментировать строки
	
	\verb*|\RequirePackage[math]{pscyr}|
	
	\verb*|\renewcommand{\rmdefault}{ftm}|
	
	Это включит использование пакета \verb*|pscyr|, который предварительно нужно установить на Вашу ОС. Этот пакет предоставляет также другие интересные шрифты, например шрифт ''как в старых учебниках'':
	
	 \verb*|\renewcommand{\rmdefault}{fac}|
	\item \verb*|xelatex| позволяет использовать в \verb|.tex| документах системные шрифты. Шаблон настроен таким образом, что при вёрстке им автоматически будет выбрано использование системного шрифта TNR (он должен быть установлен!).
\end{enumerate}

Также этот шаблон успешно верстается в онлайн системе overleaf (кроме режима с использованием pscyr). 

\newpage
Список команд, введёных в данном шаблоне и не являющихся стандартными для \LaTeX:
\begin{enumerate}
	\small
	\setlength\itemsep{-0.3cm}
	\item \verb*|\SetPDFmeta| --- установка метаданных pdf документа
	\item \verb*|\Organization| --- переменная для хранения названия организации
	\item \verb*|\OrganizationType| --- тип организации
	\item \verb*|\UpperOrganization| --- вышестоящая организация
	\item \verb*|\Faculty| --- название факультета
	\item \verb*|\Department| --- название каферды
	\item \verb*|\Superviser| --- фамилия и инициалы руководителя
	\item \verb*|\SuperviserDegree| --- учёная степень и звание (если есть) руководителя
	\item \verb*|\SuperviserWorkPlace| --- должность и место работы руководителя
	\item \verb*|\DepHead|  --- фамилия и инициалы зав. кафа
	\item \verb*|\DepHeadDegree| --- учёная степень и звание (если есть) зав. кафа
	\item \verb*|\DepHeadWorkPlace| --- должность и место работы зав. кафа
	\item \verb*|\RefSource| --- название файла с библиографией
	\item \verb*|\ChapterWithoutNum| --- вспомогательная команда для форматирования заголовка раздела без номера (Введение и т.п.)
	\item \verb*|\References| --- создания списка литературы
	\item \verb*|\Definitions| --- заголовок Определения с добавлением в оглавление без номера
	\item \verb*|\Abbreviations| --- заголовок Сокращения и обозначения с добавлением в оглавление без номера
	\item \verb*|\Introduction| --- заголовок Введение с добавлением в оглавление без номера
	\item \verb*|\Conclusion| --- заголовок Заключение с добавлением в оглавление без номера
	\item \verb*|\Appendix| --- заголовок Приложение с добавлением в оглавление без номера, с увеличением счётчика
\end{enumerate}